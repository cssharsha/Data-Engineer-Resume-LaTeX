%%%%%%%%%%%%%%%%%%%%%%%%%%%%%%%%%%%%%%%%%
% Twenty Seconds Resume/CV
% LaTeX Template
% Version 1.0 (14/7/16)
%
% Original author:
% Carmine Spagnuolo (cspagnuolo@unisa.it) with major modifications by
% Vel (vel@LaTeXTemplates.com) and Harsh (harsh.gadgil@gmail.com).
% Modifying for personal use by Sree Harsha (sreharshacs@gmail.com)
%
% License:
% The MIT License (see included LICENSE file)
%
%%%%%%%%%%%%%%%%%%%%%%%%%%%%%%%%%%%%%%%%%

%----------------------------------------------------------------------------------------
%	PACKAGES AND OTHER DOCUMENT CONFIGURATIONS
%----------------------------------------------------------------------------------------

\documentclass[letterpaper]{twentysecondcv} % a4paper for A4

% Command for printing skill overview bubbles
\newcommand\skills{
    ~
    \smartdiagram[bubble diagram]{
        \textbf{~~~~~~~~~~~~~~Robotics~~~~~~~~~~~~~},
        \textbf{~~~~~~~~~~Mapping~~~~~~~~~}\\\textbf{and}\\\textbf{Localization},
        \textbf{~~~~~~~~~~Calibration~~~~~~~~~~},
        \textbf{~~~~~Sensor~~~~~}\\\textbf{~~~~Integration~~~~},
        \textbf{Perception},
        \textbf{Navigation}
    }
}

% Programming skill bars
\programming{
    {Scripting Languages / 1},
    {Python / 3},
    {C++ / 5}}

% Projects text
\education{
    \textbf{MS, Electrical Engineering and Computer Science} (GPA: 3.86) \\
    Specialization: Robotics \\
    University of California, Merced \\
    Aug 2017 - Dec 2018 | Merced, California

    \textbf{BE, Computer Science} (GPA: 8.26 / 10) \\
    Siddaganga Institute of Technology \\
    2010 - 2014 | Tumkur, India
}

%----------------------------------------------------------------------------------------
%	 PERSONAL INFORMATION
%----------------------------------------------------------------------------------------
% If you don't need one or more of the below, just remove the content leaving the command, e.g. \cvnumberphone{}

\cvname{SREE\\HARSHA\\C S} % Your name
\cvjobtitle{Software Engineer, Mapping} % Job
% title/career

\cvlinkedin{/in/sree-harsha-c-s}
\cvgithub{cssharsha}
\cvnumberphone{(209) 658 6725} % Phone number
\cvmail{cssreddyharsha@gmail.com} % Email address

%----------------------------------------------------------------------------------------

\begin{document}

\makeprofile % Print the sidebar

%----------------------------------------------------------------------------------------
%	 EXPERIENCE
%----------------------------------------------------------------------------------------

\section{Experience}

\begin{twenty} % Environment for a list with descriptions
    \twentyitem
    {July 2021 -}
    {Present}
    {Software Engineer, Mapping and Sensor Systems}
    {\href{https://www.optimusride.com//}{Optimus Ride}}
    {}
    {\begin{itemize}
        \item Parallelized the algorithm for detecting board corners leading to a speed up of 2x-5x
            for different resolution lidars
        \item Refactor mapping tools for better analysis and visualization of the mapping stack providing easier
            modification and updation of the factor graph during map bring up
        \item POC for online out of calibration detection. Used a combination of cost functions to determine
            better perturbation baselines.
        \item \textbf{Used}: GTSAM, C++, python, bitbucket
    \end{itemize}}
    \\
    \twentyitem
    {Mar 2019 -}
    {June 2021}
    {Software Engineer}
    {\href{http://www.bossanova.com/}{Bossanova Robotics}}
    {}
    {
        {\begin{itemize}
            \item \textbf{Team}: Next-gen
                {\begin{itemize}
                    \item Integration of ToF camera and 2D lidars. Wrote ROS wrappers around the SDK along with identifying,
                        interfacing and debugging the USB device driver provided by the vendor. Developed golden dataset along with
                        metrics to help chose the sensor for next generation robots.
                    \item Porting and initial setup of the navigation stack for next generation robots. Involved modifications to move base
                        in order to allow the robot to navigate at a closer distance to the shelf and obstacles. Integrated the new sensor
                        suite into the navigation stack.
                \end{itemize}}
            \item \textbf{Team}: Perception
                {\begin{itemize}
                    \item Designed and implemented sunlight filter to detect and filter noise generated due to sunny spots in ToF camera.
                    \item Developed intrinsic calibration procedure for rgb cameras modelled as EUCM model.
                    \item Built a factor graph based 6 DoF calibration system for navigation sensors with april tags as landmarks.
                \end{itemize}}
            \item \textbf{Team}: Mapping
                {\begin{itemize}
                    \item Implemented an algorithm to perform automatic detection and updation of aisle definitions following
                        a map update
                    \item Fine-tune cartographer to generate maps using the sensor stack in the next generation of robots.
                    \item Setup cartographer in online localization mode and develop an EKF to fuse the pose estimate from cartographer
                        scan matcher and odometry.
                \end{itemize}}
            \item \textbf{Used}: GTSAM, Ceres, ROS, C++, python, git, cmake, catkin, docker
        \end{itemize}}
    }
    \\
    \twentyitem
    {Jul 2014 -}
    {Jun 2017}
    {Software Engineer}
    {Alcatel-Lucent (Now Nokia)}
    {}
    {
        \begin{itemize}
            \item Primary worked on developing platform software to provide high availability of ATCA chasis nodes.
            \item Worked on various OS related issues such as deadlocks, memory utilization and OS patching.
        \end{itemize}
        }

        %\twentyitem{<dates>}{<title>}{<location>}{<description>}
\end{twenty}

%----------------------------------------------------------------------------------------
%	 PROJECTS
%----------------------------------------------------------------------------------------
\section{Projects}
\begin{twenty}
    \twentyitem
    {2017 - 2018}
    {}
    {MS. Candidate}
    {\href{https://www.ucmerced.edu/}{University of California, Merced}}
    {}
    {
        {\begin{itemize}
            \item Designed and developed ROS nodes to detect topological features such as left/right intersection and T-intersection in an indoor environment and identify the kind of intersections along with a local planner to navigate to the interest point of the intersection.
            \item Using the parametric Bezier curve equation, render a UTAH tea pot with given control points. The normal to the surface of the teapot is also generated which is used for shading.
            \item Implemented an IK algorithm(FABRIK) to simulate the movement of a chain with 3 joints and four links.
        \end{itemize}}
    }
\end{twenty}

\end{document}
